% Whisker barrel generation paper equations, for pasting into Google
% docs draft.

% While writing, don't stop for errors
\nonstopmode

% Use the article doc class, with an 11 pt basic font size
\documentclass[11pt, a4paper]{article}

% Makes the main font Nimbus Roman, a Times New Roman lookalike:
%\usepackage{mathptmx}% http://ctan.org/pkg/mathptmx
% OR use this for proper Times New Roman (from msttcorefonts package
% on Ubuntu). Use xelatex instead of pdflatex to compile:
\usepackage{fontspec}
\usepackage{xltxtra}
\usepackage{xunicode}
\defaultfontfeatures{Scale=MatchLowercase,Mapping=tex-text}
\setmainfont{Times New Roman}

% Set margins
\usepackage[margin=2.5cm]{geometry}

% Multilingual support
\usepackage[english]{babel}

% Nice mathematics
\usepackage{amsmath}
% BOLD mathematics
\usepackage{bm}

% Left right harpoons for kinetic equations
\usepackage{mathtools}

% Control over maketitle
\usepackage{titling}

% Section styling
\usepackage{titlesec}

% Ability to use colour in text
\usepackage[usenames]{color}

% For the \degree symbol
\usepackage{gensymb}

% Allow includegraphics and nice wrapped figures
\usepackage{graphicx}
\usepackage{wrapfig}
\usepackage[outercaption]{sidecap}

% Set formats using titlesec
\titleformat*{\section}{\bfseries\rmfamily}
\titleformat*{\subsection}{\bfseries\itshape\rmfamily}

% thetitle is the number of the section. This sets the distance from
% the number to the section text.
\titlelabel{\thetitle.\hskip0.3em\relax}

% Set title spacing with titlesec, too.  The first {1.0ex plus .2ex
% minus .7ex} sets the spacing above the section title. The second
% {-1.0ex plus 0.2ex} sets the spacing the section title to the
% paragraph.
\titlespacing{\section}{0pc}{1.0ex plus .2ex minus .7ex}{-1.1ex plus 0.2ex}

%% Trick to define a language alias and permit language = {en} in the .bib file.
% From: http://tex.stackexchange.com/questions/199254/babel-define-language-synonym
\usepackage{letltxmacro}
\LetLtxMacro{\ORIGselectlanguage}{\selectlanguage}
\makeatletter
\DeclareRobustCommand{\selectlanguage}[1]{%
  \@ifundefined{alias@\string#1}
    {\ORIGselectlanguage{#1}}
    {\begingroup\edef\x{\endgroup
       \noexpand\ORIGselectlanguage{\@nameuse{alias@#1}}}\x}%
}
\newcommand{\definelanguagealias}[2]{%
  \@namedef{alias@#1}{#2}%
}
\makeatother
\definelanguagealias{en}{english}
\definelanguagealias{eng}{english}
%% End language alias trick

%% Any aliases here
\newcommand{\mb}[1]{\mathbf{#1}} % this won't work?
% Emphasis and bold.
\newcommand{\e}{\emph}
\newcommand{\mycite}[1]{\cite{#1}}
\newcommand{\code}[1]{\textsf{#1}}
\newcommand{\dvrg}{\nabla\vcdot\nabla}
%% END aliases

% Custom font defs
% fontsize is \fontsize{fontsize}{linespacesize}
\def\authorListFont{\fontsize{11}{11} }
\def\corrAuthorFont{\fontsize{10}{10} }
\def\affiliationListFont{\fontsize{11}{11}\itshape }
\def\titleFont{\fontsize{14}{11} \bfseries }
\def\textFont{\fontsize{11}{11} }
\def\sectionHdrFont{\fontsize{11}{11}\bfseries}
\def\bibFont{\fontsize{10}{10} }
\def\captionFont{\fontsize{10}{10} }

% Custom colours
\definecolor{eqgreen}{rgb}{0.220, 0.463, 0.114}
\definecolor{eqblue}{rgb}{0.235, 0.471, 0.847}
\definecolor{eqred}{rgb}{1.0, 0.0, 0.0}
\definecolor{eqgrey}{rgb}{0.8, 0.8, 0.8}
\definecolor{eqcomp}{rgb}{0.0, 0.0, 0.0}

% Caption font size to be small.
\usepackage[font=small,labelfont=bf]{caption}

% Make a dot for the dot product, call it vcdot for 'vector calculus
% dot'. Bigger than \cdot, smaller than \bullet.
\makeatletter
\newcommand*\vcdot{\mathpalette\vcdot@{.35}}
\newcommand*\vcdot@[2]{\mathbin{\vcenter{\hbox{\scalebox{#2}{$\m@th#1\bullet$}}}}}
\makeatother

\def\firstAuthorLast{James}

% Affiliations
\def\Address{\\
\affiliationListFont Adaptive Behaviour Research Group, Department of Psychology,
  The University of Sheffield, Sheffield, UK \\
}

% The Corresponding Author should be marked with an asterisk. Provide
% the exact contact address (this time including street name and city
% zip code) and email of the corresponding author
\def\corrAuthor{Seb James}
\def\corrAddress{Department of Psychology, The University of Sheffield,
  Western Bank, Sheffield, S10 2TP, UK}
\def\corrEmail{seb.james@sheffield.ac.uk}

% Figure out the font for the author list..
\def\Authors{\authorListFont Sebastian James\\[1 ex]  \Address \\
  \corrAuthorFont $^{*}$ Correspondence: \corrEmail}

% No page numbering please
\pagenumbering{gobble}

% A trick to get the bibliography to show up with 1. 2. etc in place
% of [1], [2] etc.:
\makeatletter
\renewcommand\@biblabel[1]{#1.}
\makeatother

% reduce separation between bibliography items if not using natbib:
\let\OLDthebibliography\thebibliography
\renewcommand\thebibliography[1]{
  \OLDthebibliography{#1}
  \setlength{\parskip}{0pt}
  \setlength{\itemsep}{0pt plus 0.3ex}
}

% Set correct font for bibliography (doesn't work yet)
%\renewcommand*{\bibfont}{\bibFont}

% No paragraph indenting to match the VPH format
\setlength{\parindent}{0pt}

% Skip a line after paragraphs
\setlength{\parskip}{0.5\baselineskip}
\onecolumn

% titling definitions
\pretitle{\begin{center}\titleFont}
\posttitle{\par\end{center}\vskip 0em}
\preauthor{ % Fonts are set within \Authors
        \vspace{-1.1cm} % Bring authors up towards title
        \begin{center}
        \begin{tabular}[t]{c}
}
\postauthor{\end{tabular}\par\end{center}}

% Define title, empty date and authors
\title {
  Whisker barrel paper equations
}
\date{} % No date please
\author{\Authors}

%% END OF PREAMBLE

\begin{document}
\setlength{\droptitle}{-1.8cm} % move the title up a suitable amount
\maketitle

\vspace{-1.8cm} % HACK bring the introduction up towards the title. It
                % would be better to do this with titling in \maketitle

\section{Karbowski system with N TCs and M gradients}
%
\begin{equation} \label{eq:Karb2D_dc}
\frac{\partial \color{eqgreen}\boldsymbol{c}_i(\mb{x},t)\color{black}}{\partial t} = -\alpha \color{eqgreen}\boldsymbol{c}_i(\mb{x},t)\color{black} + \beta \color{eqred}\boldsymbol{n}(\mb{x},t)\color{black}
[\color{eqblue}\boldsymbol{a}_i(\mb{x},t)\color{black}]^k
\end{equation}
%
\begin{equation} \label{eq:Karb2D_conserve}
\color{eqred}\boldsymbol{n}\color{black} = 1 - \sum_{i=1}^{N} \color{eqgreen}\boldsymbol{c}_i\color{black}
\end{equation}
%
\begin{equation} \label{eq:Karb2D_da}
\frac{\partial \color{eqblue}\boldsymbol{a}_i\color{black}}{\partial t}
= \nabla\vcdot\color{eqblue}\mb{J}_i\color{black} - \frac{\partial \color{eqgreen}\boldsymbol{c}_i\color{black}}{\partial t}
\end{equation}
with the flux current
\begin{equation} \label{eq:Karb2D_J}
\color{eqblue}\mb{J}_i(\mb{x},t)\color{black} = D \nabla \color{eqblue}\boldsymbol{a}_i\color{black} - \color{eqblue}\boldsymbol{a}_i\color{black}
\sum_{j=1}^M \big(\gamma_{i,j} \nabla\rho_j(\mb{x}) \big)
\end{equation}
%
The boundary condition applied to these equations is:
%
\begin{equation}
\color{eqblue}\mb{J}_i(\mb{x},t)\color{black} \bigg\rvert_{boundary} = 0
\end{equation}

\section{Karbowski system with N TCs and M gradients plus competition}
%
\begin{equation} \label{eq:Karb2DExt_dc}
\color{eqgrey}
\frac{\partial \color{eqgreen}\boldsymbol{c}_i\color{eqgrey}}{\partial t} = -\alpha \color{eqgreen}\boldsymbol{c}_i\color{eqgrey} + \beta \color{eqred}\boldsymbol{n}\color{eqgrey}
[\color{eqblue}\boldsymbol{a}_i\color{eqgrey}]^k
\color{black}
\end{equation}
%
\begin{equation} \label{eq:Karb2DExt_conserve}
\color{eqgrey}
\color{eqred}\boldsymbol{n}\color{eqgrey} = 1
- \sum_{i=1}^{N} \color{eqgreen}\boldsymbol{c}_i\color{eqgrey}
\color{black}
\end{equation}
%
\begin{equation} \label{eq:Karb2DExt_da}
\color{eqgrey}
\frac{\partial \color{eqblue}\boldsymbol{a}_i\color{eqgrey}}{\partial t}
= \nabla\vcdot\color{eqblue}\mb{J}_i\color{eqgrey}
- \frac{\partial \color{eqgreen}\boldsymbol{c}_i\color{eqgrey}}{\partial t}
\color{eqcomp}
- \frac{\epsilon \color{eqblue}\boldsymbol{a}_i}{\color{eqcomp}N-1} \sum_{j \ne i}^{N} \color{eqblue}\boldsymbol{a}_j\color{eqcomp}^l
\color{black}
\end{equation}
%
And normalization step

\begin{equation}\label{eq:norm_comp8}
\color{eqcomp}
\color{eqblue}\boldsymbol{a}_i'\color{eqcomp} = n_e \; \frac {\color{eqblue}\boldsymbol{a}_i\color{eqcomp}} {\textstyle \sum_j^{n_e} \color{eqblue}\boldsymbol{a}_{i,j}\color{eqcomp} }
\color{black}
\end{equation}
%
($a_i'$ becomes $a_i$ in the next computational step.)
%
\end{document}
